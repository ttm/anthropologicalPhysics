%%%%%%%%%%%%%%%%%%%%%%%%%%%%%%%%%%%%%%%%%
% Thin Sectioned Essay
% LaTeX Template
% Version 1.0 (3/8/13)
%
% This template has been downloaded from:
% http://www.LaTeXTemplates.com
%
% Original Author:
% Nicolas Diaz (nsdiaz@uc.cl) with extensive modifications by:
% Vel (vel@latextemplates.com)
%
% License:
% CC BY-NC-SA 3.0 (http://creativecommons.org/licenses/by-nc-sa/3.0/)
%
%%%%%%%%%%%%%%%%%%%%%%%%%%%%%%%%%%%%%%%%%

%----------------------------------------------------------------------------------------
%   PACKAGES AND OTHER DOCUMENT CONFIGURATIONS
%----------------------------------------------------------------------------------------

\documentclass[a4paper, 11pt]{article} % Font size (can be 10pt, 11pt or 12pt) and paper size (remove a4paper for US letter paper)
\usepackage{hyperref}
\hypersetup{
        colorlinks,
            linkcolor={red!50!black},
                citecolor={blue!50!black},
                    urlcolor={blue!80!black}
                }


\usepackage[portuguese,english]{babel}
\usepackage[utf8]{inputenc}
%\usepackage{float}

%\usepackage{color} % for the notes
\usepackage{xcolor}
\usepackage[protrusion=true,expansion=true]{microtype} % Better typography
%\usepackage{graphicx} % Required for including pictures
%\usepackage{wrapfig} % Allows in-line images
%\usepackage{tocloft}
%\usepackage{multirow}

\usepackage{mathpazo} % Use the Palatino font
%\usepackage[T1]{fontenc} % Required for accented characters
%\linespread{1.05} % Change line spacing here, Palatino benefits from a slight increase by default
%\usepackage{etoolbox}
\newcommand{\githubi}{Git\textsc{h}ub}
\newcommand{\bdoh}{{\sc b}lack {\sc d}uck {\sc o}pen \textsc{hub}}
\newcommand{\ohloh}{\textsc{o}hloh}
\newcommand{\php}{\textsc{php}}
\newcommand{\twitter}{\textsc{t}witter}
\newcommand{\facebook}{\textsc{f}acebook}
\newcommand{\msn}{\textsc{msn}}
\newcommand{\gchat}{\textsc{g}oogle \textsc{c}hat}
\newcommand{\bash}{\textsc{b}ash}
\newcommand{\python}{\textsc{p}ython}
\newcommand{\django}{\textsc{d}jango}
\newcommand{\curl}{c\textsc{url}}
\newcommand{\firefox}{\textsc{f}irefox}
\newcommand{\floss}{\textsc{floss}}
\newcommand{\openoffice}{\textsc{o}pen\textsc{o}ffice}
\newcommand{\puredata}{\textsc{p}uredata}
\newcommand{\schema}{\textsc{s}chema.org}
\newcommand{\wiki}{\textsc{w}iki}
\newcommand{\nosql}{\textsc{n}o\textsc{sql}}
\newcommand{\etherpad}{\textsc{e}therpad}
\newcommand{\irc}{\textsc{irc}}
\newcommand{\irci}{\textsc{Irc}}
\newcommand{\ocd}{\textsc{ocd}}
\newcommand{\participa}{\textsc{p}articipa.br}
\newcommand{\httpb}{\textsc{http}}
\newcommand{\foaf}{\textsc{foaf}}
\newcommand{\ops}{\textsc{ops}}
\newcommand{\sioc}{\textsc{sioc}}
\newcommand{\gndo}{\textsc{gndo}}
\newcommand{\html}{\textsc{html}}
\newcommand{\ggg}{\textsc{ggg}}
\newcommand{\opa}{\textsc{opa}}
\newcommand{\obs}{\textsc{obs}}
\newcommand{\vbs}{\textsc{vbs}}
\newcommand{\lod}{\textsc{lod}}
\newcommand{\nlp}{\textsc{nlp}}
\newcommand{\sectionb}{\textsc{s}ection}
\newcommand{\cn}{\textsc{cn}}
\newcommand{\aab}{\textsc{aa}}
\newcommand{\dc}{\textsc{d}ublin {\sc c}ore}
\newcommand{\json}{\textsc{json}}
\newcommand{\flask}{\textsc{f}lask}
\newcommand{\aai}{\textsc{Aa}}
\newcommand{\ontologiaa}{\textsc{o}ntologi\textsc{aa}}
\newcommand{\ontologiaai}{\textsc{O}ntologi\textsc{aa}}
\newcommand{\owl}{{\sc owl}}
\newcommand{\www}{{\sc www}}
\newcommand{\rdfi}{{\sc Rdf}}
\newcommand{\mongodb}{{\sc m}ongo{\sc db}}
\newcommand{\mysql}{{\sc m}y{\sc sql}}
\newcommand{\rdf}{{\sc rdf}}
%\newcommand{\paaineli}{P{\sc aa}inel}
\newcommand{\paaineli}{P{\bf \sc aa}inel}
\newcommand{\paainel}{p{\sc aa}inel}
\newcommand{\gsd}{\textsc{gsd}}
\newcommand{\ui}{\textsc{ui}}
%\newcommand{\lmb}{\url{lab\textsc{M}acambira.sf.net}}
\newcommand{\lm}{lab\textsc{M}acambira.sf.net}
%\newcommand{\lm}{\url{labMacambira.sf.net}}



\makeatletter
\renewcommand\@biblabel[1]{\textbf{#1.}} % Change the square brackets for each bibliography item from '[1]' to '1.'
\renewcommand{\@listI}{\itemsep=0pt} % Reduce the space between items in the itemize and enumerate environments and the bibliography

\usepackage{epigraph}
%\pretocmd{\chapter}{\addtocontents{toc}{\protect\addvspace{5\p@}}}{}{}
%\pretocmd{\section}{\addtocontents{toc}{\protect\vspace{-4mm}}}{}{}
\renewcommand{\maketitle}{ % Customize the title - do not edit title and author name here, see the TITLE block below
\begin{flushright} % Right align
{\LARGE\@title} % Increase the font size of the title

\vspace{50pt} % Some vertical space between the title and author name

{\large\@author} % Author name
\\\@date % Date

\vspace{40pt} % Some vertical space between the author block and abstract
\end{flushright}
}

%----------------------------------------------------------------------------------------
%   TITLE
%----------------------------------------------------------------------------------------

\title{\textbf{What are you and I?}\\ % Title
%a natural collective focus\\on the collective being} % Subtitle
anthropological physics fundamentals} % Subtitle

\author{\textsc{Renato Fabbri} % Author
\\{\textit{IFSC/USP, Participa.br/SG-PR, labMacambira.sf.net}}} % Institution

\date{\today} % Date

%----------------------------------------------------------------------------------------

\begin{document}

\maketitle % Print the title section

%----------------------------------------------------------------------------------------
%   ABSTRACT AND KEYWORDS
%----------------------------------------------------------------------------------------

%\renewcommand{\abstractname}{Summary} % Uncomment to change the name of the abstract to something else

%
%\begin{abstract}
%\end{abstract}
%
%{
%\selectlanguage{portuguese}
%\begin{abstract}
%
%\end{abstract}
%}

\hspace*{3,6mm}\textit{Keywords:} anthropological physics, complex systems, complex networks, natural language processing, social network analysis, semantic web, social participation, ethnographic methods, \floss
%, statistics % Keywords

%\vspace{30pt} % Some vertical space between the abstract and first section

%----------------------------------------------------------------------------------------
%   ESSAY BODY
%----------------------------------------------------------------------------------------
%\newpage
%\tableofcontents
\vspace*{1cm}
{\bf This is a mere report on the newborn concept of \emph{anthropological physics}. Further efforts should contextualize, develop and correct theoretical nuances. Therefore, the sharing of this naive text is a necessary step to the collective maturing and research.}
\vspace*{.6cm}

\newpage
\epigraph{A single dramatic incident involving a breach of privacy could produce a set of statutes, rules, and prohibitions that could strangle the nascent field of computational social science in its crib. What is necessary, now, is to produce a self-regulatory regime of procedures, technologies, and rules that reduce this risk but preserve most of the research potential.}{David Lazer, Alex (Sandy) Pentland, Lada Adamic, Sinan Aral, Albert Laszlo Barabasi, Devon Brewer, Nicholas Christakis, Noshir Contractor, James Fowler, Myron Gutmann, Tony Jebara, Gary King, Michael Macy, Deb Roy, and Marshall Van Alstyne~\cite{life}}

\section*{What}
The study of complex systems can be undertaken as a physics endeavor, specially if complex networks and statistics are into play.
%The study of humans is, by excellence, anthropology. 
When the complex system is constituted by people, intriguing questions arise from diverse field such as math, ethics, and sociology. The ``anthropological physics'' is an approach to these scenarios that enables scientific research while resolving ethical and moral issues by an open study of the self.

\section*{How}
If I annotate what I have been doing and what is happening in my environment, I might be able to use those annotations for my studies.
Indeed, writing diaries is a common ethnographic method, used often in scientific research. Thereafter, it might be reasonable
to use my own annotations, be them in paper diaries or digital media~\cite{wolfram}. It is also reasonable that if a partner wants to use his annotations, or want me to use them, that they be used. There are some initiatives that receive this kind of data, again evoking ethical and moral issues. A sweet spot was found in recent civil, research and academic efforts~\cite{pnud5,ensaio}, and has been called anthropological physics. General characteristics are:

\begin{itemize}
    \item Exposure of the researcher to the environment of interest, such as virtual social networks.
    \item Use of the annotations from the exposure, be them activity logs, friendship or interaction networks, textual contents, etc.
    \item Upon need, expansion of observations to open datasets or data donated by partners.
    \item All resources are kept as open and publicized as possible, including software, data, and writings.
\end{itemize}

This framework made possible endeavors using \facebook, \twitter, email lists, and alternative networks~\cite{ars,ocupagov,stabNet,textNet,pnud5}, while maintaining ethical agreement among communities and researchers.

\section*{With}
The open aspect of anthropological physics eased technological support, of which are noteworthy:
\begin{itemize}
    \item Linked data/semantic web and other open standards for data. As data is being used by research, if it is not considered invasive, it might be published as \rdf\ triples and related to \owl\ ontologies~\cite{LOD,linkedDataBook,pnud5}.
    \item Resource exploitation for the individual by complex networks and natural language criteria. This is envisioned as highly serviceable to individuals and collectives, and as a linked data navigation enhancement~\cite{pnud3,pnud4}.
    \item Extensive use of Free, Libre and Open Source Software (\floss). This helps results and procedures to be shared by means of immediate access to the tools, versions, and underlying algorithms.
    \item Social structures streaming. Real-time exposure of our networks is called upon for transparency of public events~\cite{ocupagov}.
\end{itemize}

\section*{When}
Application of anthropological physics gives place in everyday research worldwide.
The explicit (spoken) use of the concept has been observed in Brazilian academic circuits since
Feb/2013, with contributions by physicists, anthropologists, social scientists and philosophers.
Even so, this text is the first written document for interested community reviews.
Feedback should yield significant changes to this content. The academic
community is presenting articulations that support further maturing of the topic~\cite{reunioesNexus}.

Current efforts are oriented to: presenting 
a coherent linked data legacy of participative data 
as journal articles~\cite{pnud5}; implementing
resource exploitation techniques for the individual~\cite{pnud4};
 implementing visualization facilities for sharing useful insights
and report~\cite{ocupagov,appGMANE}.

%\section{\aai\ concept}\label{sec:start}
\subsection*{Acknowledgments}
Author thank Prof. Dr. Massimo Canevacci (IEA/USP), Marília Pisani (CCNH/UFABC), 
Deborah Antunes (Psychology, UFC), Cassia Wu, Juliana de Souza, and Pedro (Attraktor) Rocha for invaluable insights
and practice opportunities; Ricardo Fabbri, Vilson Vieira, Daniel Penalva, Edson Corrêa Jr. and all
labMacambira.sf.net members for pursuing this and other developments;
the General Secretariat of the Republic Presidency (SG-PR) and UNDP for supporting this
research (contract 2013/00056, project BRA/12/018); the National Counsel of Technological 
and Scientific Development (process 140860/2013-4, project 870336/1997-5,
advisor: Prof. Dr. Osvaldo Novais de Oliveira Junior);
to all authors of the inspiring article~\cite{life} which encouraged this present essay
and is the source of the epigraph.

%\addcontentsline{toc}{subsection}{\ontologiaa: the \aab\ ontology}
%\begin{wrapfigure}{l}{0.4\textwidth} % Inline image example
%\begin{center}
%\includegraphics[width=0.38\textwidth]{telao1.png}
%\end{center}
%\caption{\small Telão para streaming de estruturas sociais, usado no \#arenaNETmundial, \#ocupaGOV e outras ocasiões. Tela com rede de retweets e relacionamento via hashtag e vocabulário. Atualizada a cada 10 segundos com os relacionamentos implicados pelos dos tweets mais recentes.}\label{fig:telao}
%\end{wrapfigure}

%\begin{figure}[H]
%  \centering
%    \includegraphics[width=.7\textwidth]{telao2.png}
%  \caption{\small Telão para streaming de estruturas sociais, usado no \#arenaNETmundial, \#ocupaGOV e outras ocasiões. Tela com relacionamentos de hashtags e vocabulário. Atualizada a cada 10 segundos com conteúdo dos tweets mais recentes.}\label{fig:telao2}
%\end{figure}










%----------------------------------------------------------------------------------------
%   BIBLIOGRAPHY
%----------------------------------------------------------------------------------------

%\bibliographystyle{unsrt}
%\bibliographystyle{plain}
\bibliographystyle{ieeetr}
\bibliography{essay}

%----------------------------------------------------------------------------------------

\end{document}
